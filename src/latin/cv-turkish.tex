\documentclass[10pt,a4paper]{article}

% Encoding and standard fonts (ATS-friendly)
\usepackage[utf8]{inputenc}
\usepackage[T1]{fontenc}
\usepackage{lmodern}
\renewcommand{\familydefault}{\sfdefault}

% Compact layout for 2 pages
\usepackage[a4paper,margin=1.2cm,top=1.3cm,bottom=1.2cm]{geometry}
\setlength{\parindent}{0pt}
\setlength{\parskip}{1.5pt}
\linespread{1.05}

% Hyperlinks + metadata (ATS-friendly)
\usepackage[hidelinks]{hyperref}
\hypersetup{
  pdftitle={Utku Cikmaz – Full Stack Developer},
  pdfauthor={Utku Cikmaz},
  pdfsubject={Resume, CV, Utku Cikmaz, Full Stack Developer},
  pdfkeywords={Full Stack Developer, React, Vue, JavaScript, Python, Django, Frontend, Backend, AI, Data Visualization, User Experience, Accessibility, Test-Driven Development, Performance Optimization, Cloud Deployment, CI/CD, Git, Docker, Webpack, Vite, Jest, Cypress, Vitest, Supertest, Test Coverage, Code Quality, Agile Methodologies, Scrum, RESTful APIs, Database Design, Microservices, Role-Based Access Control, Multi-Tenant Applications},
  pdfproducer={LaTeX},
  pdfcreator={LaTeX}
}

% Minimal grayscale color palette
\usepackage{xcolor}
\definecolor{darkgray}{HTML}{2D2D2D}     % Dark gray
\definecolor{mediumgray}{HTML}{5A5A5A}   % Medium gray
\definecolor{lightgray}{HTML}{8A8A8A}    % Light gray
\definecolor{verylightgray}{HTML}{B0B0B0} % Very light gray
\definecolor{black}{HTML}{000000}        % Black

% Gradient support for name
\usepackage{tikz}
\usetikzlibrary{shapes,positioning,calc}

% Simple list styling
\usepackage{enumitem}
\setlist[itemize]{
  leftmargin=10pt,
  topsep=1.5pt,
  itemsep=1pt,
  parsep=0pt,
  label=\textcolor{mediumgray}{$\bullet$},
  labelsep=5pt
}

% Clean section styling
\usepackage{titlesec}
\titleformat{\section}
  {\color{darkgray}\large\bfseries\sffamily}
  {}
  {0pt}
  {}
  [\vspace{-4pt}\textcolor{verylightgray}{\titlerule[0.8pt]}\vspace{3pt}]

\titlespacing{\section}{0pt}{8pt}{4pt}

% Prevent section breaks
\usepackage{needspace}

% Compact custom commands
\newcommand{\Experience}[4]{%
  \vspace{2pt}
  \noindent
  \begin{minipage}[t]{0.70\textwidth}
    {\normalsize\textbf{\color{darkgray}#3}} \\
    \textcolor{lightgray}{\small #1}
  \end{minipage}%
  \hfill
  \begin{minipage}[t]{0.28\textwidth}
    \raggedleft
    \textcolor{mediumgray}{\small\textbf{#4}} \\
    \textcolor{lightgray}{\small #2}
  \end{minipage}
  \vspace{1pt}
}

\newcommand{\Education}[3]{%
  \vspace{2pt}
  \noindent
  \begin{minipage}[t]{0.70\textwidth}
    {\normalsize\textbf{\color{darkgray}#1}} \\
    \textcolor{lightgray}{\small #2}
  \end{minipage}%
  \hfill
  \begin{minipage}[t]{0.28\textwidth}
    \raggedleft
    \textcolor{mediumgray}{\small\textbf{#3}}
  \end{minipage}
  \vspace{1pt}
}

\newcommand{\Project}[2]{%
  \vspace{2pt}
  \noindent{\normalsize\textbf{\color{darkgray}#1}} \\
  \textcolor{lightgray}{\small #2}
  \vspace{1pt}
}

% Simple gradient name using varying shades
\newcommand{\gradientname}[1]{%
  \textcolor{darkgray!90}{\Huge\bfseries U}%
  \textcolor{darkgray!80}{\Huge\bfseries T}%
  \textcolor{darkgray!70}{\Huge\bfseries K}%
  \textcolor{darkgray!60}{\Huge\bfseries U}%
  \textcolor{mediumgray!90}{\Huge\bfseries\ }%
  \textcolor{mediumgray!80}{\Huge\bfseries C}%
  \textcolor{mediumgray!70}{\Huge\bfseries I}%
  \textcolor{mediumgray!60}{\Huge\bfseries K}%
  \textcolor{mediumgray!50}{\Huge\bfseries M}%
  \textcolor{lightgray!80}{\Huge\bfseries A}%
  \textcolor{lightgray!70}{\Huge\bfseries Z}%
}

\begin{document}
\color{black}
\pagestyle{empty}

% Minimal header with gradient name
\begin{center}
  \vspace{-3pt}
  \gradientname{UTKU CIKMAZ}
  
  \vspace{4pt}
  \textcolor{verylightgray}{\rule{0.35\textwidth}{0.6pt}}
  
  \vspace{4pt}
  {\large\color{mediumgray}Tam Yığın Geliştirici}
  
  \vspace{5pt}
  \begin{tabular}{c}
    \textcolor{lightgray}{\small Hamburg, Germany} \\[2pt]
    \href{mailto:utkucikmaz@gmail.com}{\textcolor{darkgray}{utkucikmaz@gmail.com}} \quad
    \textcolor{verylightgray}{$\bullet$} \quad
    \href{https://www.utkucikmaz.com}{\textcolor{darkgray}{utkucikmaz.com}} \\[1pt]
    \href{https://linkedin.com/in/utkucikmaz}{\textcolor{darkgray}{linkedin.com/in/utkucikmaz}} \quad
    \textcolor{verylightgray}{$\bullet$} \quad
    \href{https://github.com/utkucikmaz}{\textcolor{darkgray}{github.com/utkucikmaz}}
  \end{tabular}
\end{center}

\vspace{1pt}

\section*{Profesyonel Profil}
\vspace{0pt}
\textcolor{darkgray}{%
Sonuç odaklı Tam Yığın Geliştirici, ölçeklenebilir, yüksek performanslı web uygulamaları oluşturmada kanıtlanmış uzmanlığa sahip. Modern JavaScript ekosistemlerinde (React, Vue, Next.js) ve sağlam backend mimarilerinde (Django, Node.js, NestJS) uzmanlaşmış. Teknik migrasyonları yönetme, kod tabanlarını optimize etme ve üretime hazır çözümler sunmada gösterilen başarı. Kod kalitesi, test odaklı geliştirme ve erişilebilirlik standartlarının güçlü savunucusu. Şu anda veri görselleştirme ve kullanıcı deneyimi mükemmelliğine odaklanarak AI destekli platformlar geliştiriyor.
}
\vspace{1pt}

\needspace{8\baselineskip}
\section*{Profesyonel Deneyim}

\Experience{AIlon / ERASON}{Lüneburg, Germany}{Tam Yığın Geliştirici}{Eki 2024 -- Günümüz}
\begin{itemize}
  \item Vue 3, Django REST Framework ve Highcharts kullanarak kurumsal müşterilere hizmet veren üretim seviyesinde AI destekli analitik platform mimarisi ve geliştirme
  \item Büyük ölçekli veri setlerini işleyen yüksek performanslı veri görselleştirme bileşenleri mühendisliği, optimizasyon teknikleriyle render süresini \%60 azaltma
  \item Platform genelinde görsel tutarlılığı sağlayan kapsamlı tasarım sistemi ve bileşen kütüphanesi oluşturma, UI geliştirme süresini \%40 azaltma
  \item Vue 3 Composition API, Pinia ve TanStack Query kullanarak gelişmiş durum yönetimi mimarisi uygulama, uygulama yanıt hızını iyileştirme
  \item Ürün ve tasarım ekipleriyle stratejik kararlarda işbirliği, yol haritası planlamasına ve teknik fizibilite değerlendirmelerine katkıda bulunma
\end{itemize}

\Experience{Arbithub}{Chicago, USA (Remote)}{Ön Uç Web Geliştirici Stajyeri}{Oca 2024 -- Ağu 2024}
\begin{itemize}
  \item Sıfır kesinti süresiyle kritik framework migrasyonu (React 17→18, React Router v5→v6, Redux Toolkit, Tailwind CSS v2→v3) öncülük etme, bundle boyutunu \%25 iyileştirme ve geliştirici deneyimini artırma
  \item Gelişmiş filtreleme, debouncing ve gerçek zamanlı güncellemelerle kurumsal seviye arama işlevselliği oluşturma, kullanıcı katılımını \%35 artırma
  \item ESLint ve Prettier ile kod kalitesi standartları oluşturma, teknik borcu \%40 azaltma ve ekip genelinde geliştirme iş akışını standartlaştırma
  \item Agile ortamında özellikler sunma, çapraz fonksiyonel ekiplerle günlük standuplar, sprint planlaması ve kod incelemelerine katılma
\end{itemize}

\Experience{LightUp Digital}{Berlin, Germany (Remote)}{Tam Yığın Geliştirici Stajyeri}{Eki 2023 -- Oca 2024}
\begin{itemize}
  \item TypeScript, bağımlılık enjeksiyonu ve modüler tasarım desenleriyle NestJS kullanarak ölçeklenebilir mikroservis mimarisi tasarlama ve uygulama
  \item React ve Material-UI ile üretime hazır duyarlı arayüzler geliştirme, geliştirme süresini \%30 azaltan yeniden kullanılabilir bileşen kütüphanesi oluşturma
  \item JWT ve OAuth 2.0 ile güvenli kimlik doğrulama sistemi mimarisi, çok kiracılı uygulama için rol tabanlı erişim kontrolü (RBAC) uygulama
  \item Jest ve Supertest ile \%85 kod kapsamına ulaşan kapsamlı test stratejisi oluşturma, güvenilirlik ve sürdürülebilirliği sağlama
\end{itemize}

\Experience{Turkish Airlines}{Istanbul, Turkey}{IOCC Operasyon Kontrol Subayı}{Mar 2019 -- May 2020}
\begin{itemize}
  \item Görev kritik ortamda 200+ günlük uçuşun gerçek zamanlı operasyonel kontrolünü yönetme, yüksek basınçlı karar verme ve sistem güvenilirliği gerektiren
  \item Çapraz fonksiyonel uluslararası ekipleri koordine etme, dağıtılmış yazılım geliştirmeye uygulanabilir güçlü iletişim ve problem çözme becerileri geliştirme
\end{itemize}

\needspace{6\baselineskip}
\section*{Öne Çıkan Projeler}

\Project{Hotel Middle Earth — Web Uygulama Oyunu}{%
  Sürükleyici hikaye anlatımı ve gerçek zamanlı çok oyunculu etkileşimlerle interaktif macera oyunu \textcolor{verylightgray}{|} HTML, Web Uygulamaları, Nesne Yönelimli Programlama (OOP), JavaScript, CSS. Oyun puanlama ve liderlik tablosu işlevselliği için Firebase ile backend sunucu mimarisi uygulama
}

\Project{PlantIQ — Akıllı Bitki Bakım Uygulaması}{%
  Üretime hazır bitki tanımlama ve bakım yönetim sistemi \textcolor{verylightgray}{|} JavaScript, Express.js, PlantNet API, Trefle API, Handlebars. Gerçek zamanlı bitki sağlığı takibi ve kişiselleştirilmiş bakım önerileriyle duyarlı mobil öncelikli arayüz oluşturma
}

\Project{React Movie App — Eğlence Keşif Platformu}{%
  Gelişmiş filtreleme ve kullanıcı tercihleriyle özellik açısından zengin film keşif uygulaması \textcolor{verylightgray}{|} React.js, özel hook'lar, useReducer, Firebase Auth, Supabase, Cypress E2E testleri, TMDB API. Güvenilirlik ve sürdürülebilirliği sağlayan kapsamlı test kapsamı uygulama
}

\Project{Memoria — AI Geliştirilmiş Sosyal Platform}{%
  AI destekli içerik üretimi ile tam yığın sosyal ağ platformu \textcolor{verylightgray}{|} React.js, Node.js, Express, MongoDB, OpenAI API, Tailwind CSS, JWT kimlik doğrulama. ARIA etiketleri ve klavye navigasyonu ile erişilebilirlik odaklı tasarım uygulama, WCAG 2.1 AA uyumluluğuna ulaşma
}

\needspace{5\baselineskip}
\section*{Teknik Beceriler}

\noindent
\textbf{\color{darkgray}Ön Uç \& Çerçeveler:} React.js, Vue.js, Next.js, TypeScript, JavaScript (ES6+), HTML5, CSS3 \\
\textbf{\color{darkgray}Arka Uç \& API'ler:} Python, Django, Django REST Framework, Node.js, Express.js, NestJS, RESTful APIs \\
\textbf{\color{darkgray}Veritabanları \& ORM'ler:} MongoDB, PostgreSQL, Mongoose, Supabase, Veritabanı Tasarımı \& Optimizasyon \\
\textbf{\color{darkgray}Stil \& UI:} Tailwind CSS, Sass/SCSS, Styled Components, CSS Modules, Bootstrap, Material-UI \\
\textbf{\color{darkgray}Durum Yönetimi:} Redux Toolkit, Pinia, TanStack Query, Context API, Zustand \\
\textbf{\color{darkgray}Test \& Kalite:} Jest, Cypress, Vitest, Supertest, Test Odaklı Geliştirme, Kod Kapsamı \\
\textbf{\color{darkgray}DevOps \& CI/CD:} Git, GitHub Actions, Docker, Webpack, Vite, Build Optimizasyonu \\
\textbf{\color{darkgray}Bulut \& Hizmetler:} Firebase, Supabase, Vercel, Netlify, AWS S3, Bulut Dağıtımı

\needspace{4\baselineskip}
\section*{Anahtar Yetkinlikler}

\noindent
\begin{minipage}[t]{0.48\textwidth}
  \begin{itemize}
    \item Agile \& Scrum Metodolojileri
    \item Çapraz Fonksiyonel İşbirliği
    \item Kod İnceleme \& Mentorluk
    \item Problem Çözme \& Hata Ayıklama
  \end{itemize}
\end{minipage}%
\hfill
\begin{minipage}[t]{0.48\textwidth}
  \begin{itemize}
    \item Test Odaklı Geliştirme (TDD)
    \item Performans Optimizasyonu
    \item UI/UX Tasarım İlkeleri
    \item Web Erişilebilirliği (WCAG 2.1)
  \end{itemize}
\end{minipage}

\needspace{6\baselineskip}
\section*{Eğitim}

\Education{Tam Yığın Web Geliştirme Bootcamp}{Ironhack — Uzaktan (400 saat yoğun)}{2023}

\Education{Tam Yığın Web Geliştirme Bootcamp}{Digi-Homeschooling — Uzaktan (400 saat yoğun)}{2022}

\Education{Çin Dili ve Edebiyatı Lisans Derecesi}{Istanbul University, Istanbul, Turkey}{2016 -- 2020}

\Education{Sosyoloji Lisans Derecesi}{Turkish Military Academy, Ankara, Turkey}{2012 -- 2015}

\section*{Diller}

\vspace{1pt}
\noindent
\begin{minipage}[t]{0.24\textwidth}
  \textcolor{darkgray}{\textbf{Türkçe:}} \\
  \textcolor{lightgray}{\small Ana Dil}
\end{minipage}%
\hfill
\begin{minipage}[t]{0.24\textwidth}
  \textcolor{darkgray}{\textbf{İngilizce:}} \\
  \textcolor{lightgray}{\small Tam Profesyonel}
\end{minipage}%
\hfill
\begin{minipage}[t]{0.24\textwidth}
  \textcolor{darkgray}{\textbf{Almanca:}} \\
  \textcolor{lightgray}{\small Tam Profesyonel}
\end{minipage}%
\hfill
\begin{minipage}[t]{0.24\textwidth}
  \textcolor{darkgray}{\textbf{Çince:}} \\
  \textcolor{lightgray}{\small Sınırlı}
\end{minipage}

\end{document}
