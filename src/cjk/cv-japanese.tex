\documentclass[10pt,a4paper]{article}

% Encoding and standard fonts (ATS-friendly)
\usepackage{fontspec}

% Font fallback system - try multiple fonts for compatibility
\IfFontExistsTF{Noto Sans CJK JP}{
  \setmainfont{Noto Sans CJK JP}
  \setsansfont{Noto Sans CJK JP}
}{
  \IfFontExistsTF{Hiragino Sans}{
    \setmainfont{Hiragino Sans}
    \setsansfont{Hiragino Sans}
  }{
    \IfFontExistsTF{Hiragino Kaku Gothic ProN}{
      \setmainfont{Hiragino Kaku Gothic ProN}
      \setsansfont{Hiragino Kaku Gothic ProN}
    }{
      \IfFontExistsTF{Arial Unicode MS}{
        \setmainfont{Arial Unicode MS}
        \setsansfont{Arial Unicode MS}
      }{
        \setmainfont{IPAGothic}
        \setsansfont{IPAGothic}
      }
    }
  }
}
\renewcommand{\familydefault}{\sfdefault}

% Compact layout for 2 pages
\usepackage[a4paper,margin=1.2cm,top=1.3cm,bottom=1.2cm]{geometry}
\setlength{\parindent}{0pt}
\setlength{\parskip}{1.5pt}
\linespread{1.05}

% Hyperlinks + metadata (ATS-friendly)
\usepackage[hidelinks]{hyperref}
\hypersetup{
  pdftitle={Utku Cikmaz – Full Stack Developer},
  pdfauthor={Utku Cikmaz},
  pdfsubject={Resume, CV, Utku Cikmaz, Full Stack Developer},
  pdfkeywords={Full Stack Developer, React, Vue, JavaScript, Python, Django, Frontend, Backend, AI, Data Visualization, User Experience, Accessibility, Test-Driven Development, Performance Optimization, Cloud Deployment, CI/CD, Git, Docker, Webpack, Vite, Jest, Cypress, Vitest, Supertest, Test Coverage, Code Quality, Agile Methodologies, Scrum, RESTful APIs, Database Design, Microservices, Role-Based Access Control, Multi-Tenant Applications},
  pdfproducer={LaTeX},
  pdfcreator={LaTeX}
}

% Minimal grayscale color palette
\usepackage{xcolor}
\definecolor{darkgray}{HTML}{2D2D2D}     % Dark gray
\definecolor{mediumgray}{HTML}{5A5A5A}   % Medium gray
\definecolor{lightgray}{HTML}{8A8A8A}    % Light gray
\definecolor{verylightgray}{HTML}{B0B0B0} % Very light gray
\definecolor{black}{HTML}{000000}        % Black

% Gradient support for name
\usepackage{tikz}
\usetikzlibrary{shapes,positioning,calc}

% Simple list styling
\usepackage{enumitem}
\setlist[itemize]{
  leftmargin=10pt,
  topsep=1.5pt,
  itemsep=1pt,
  parsep=0pt,
  label=\textcolor{mediumgray}{$\bullet$},
  labelsep=5pt
}

% Clean section styling
\usepackage{titlesec}
\titleformat{\section}
  {\color{darkgray}\large\bfseries\sffamily}
  {}
  {0pt}
  {}
  [\vspace{-4pt}\textcolor{verylightgray}{\titlerule[0.8pt]}\vspace{3pt}]

\titlespacing{\section}{0pt}{8pt}{4pt}

% Prevent section breaks
\usepackage{needspace}

% Compact custom commands
\newcommand{\Experience}[4]{%
  \vspace{2pt}
  \noindent
  \begin{minipage}[t]{0.70\textwidth}
    {\normalsize\textbf{\color{darkgray}#3}} \\
    \textcolor{lightgray}{\small #1}
  \end{minipage}%
  \hfill
  \begin{minipage}[t]{0.28\textwidth}
    \raggedleft
    \textcolor{mediumgray}{\small\textbf{#4}} \\
    \textcolor{lightgray}{\small #2}
  \end{minipage}
  \vspace{1pt}
}

\newcommand{\Education}[3]{%
  \vspace{2pt}
  \noindent
  \begin{minipage}[t]{0.70\textwidth}
    {\normalsize\textbf{\color{darkgray}#1}} \\
    \textcolor{lightgray}{\small #2}
  \end{minipage}%
  \hfill
  \begin{minipage}[t]{0.28\textwidth}
    \raggedleft
    \textcolor{mediumgray}{\small\textbf{#3}}
  \end{minipage}
  \vspace{1pt}
}

\newcommand{\Project}[2]{%
  \vspace{2pt}
  \noindent{\normalsize\textbf{\color{darkgray}#1}} \\
  \textcolor{lightgray}{\small #2}
  \vspace{1pt}
}

% Simple gradient name using varying shades
\newcommand{\gradientname}[1]{%
  \textcolor{darkgray!90}{\Huge\bfseries U}%
  \textcolor{darkgray!80}{\Huge\bfseries T}%
  \textcolor{darkgray!70}{\Huge\bfseries K}%
  \textcolor{darkgray!60}{\Huge\bfseries U}%
  \textcolor{mediumgray!90}{\Huge\bfseries\ }%
  \textcolor{mediumgray!80}{\Huge\bfseries C}%
  \textcolor{mediumgray!70}{\Huge\bfseries I}%
  \textcolor{mediumgray!60}{\Huge\bfseries K}%
  \textcolor{mediumgray!50}{\Huge\bfseries M}%
  \textcolor{lightgray!80}{\Huge\bfseries A}%
  \textcolor{lightgray!70}{\Huge\bfseries Z}%
}

\begin{document}
\color{black}
\pagestyle{empty}

% Minimal header with gradient name
\begin{center}
  \vspace{-3pt}
  \gradientname{UTKU CIKMAZ}
  
  \vspace{4pt}
  \textcolor{verylightgray}{\rule{0.35\textwidth}{0.6pt}}
  
  \vspace{4pt}
  {\large\color{mediumgray}フルスタック開発者}
  
  \vspace{5pt}
  \begin{tabular}{c}
    \textcolor{lightgray}{\small Hamburg, Germany} \\[2pt]
    \href{mailto:utkucikmaz@gmail.com}{\textcolor{darkgray}{utkucikmaz@gmail.com}} \quad
    \textcolor{verylightgray}{$\bullet$} \quad
    \href{https://www.utkucikmaz.com}{\textcolor{darkgray}{utkucikmaz.com}} \\[1pt]
    \href{https://linkedin.com/in/utkucikmaz}{\textcolor{darkgray}{linkedin.com/in/utkucikmaz}} \quad
    \textcolor{verylightgray}{$\bullet$} \quad
    \href{https://github.com/utkucikmaz}{\textcolor{darkgray}{github.com/utkucikmaz}}
  \end{tabular}
\end{center}

\vspace{1pt}

\section*{プロフェッショナルプロフィール}
\vspace{0pt}
\textcolor{darkgray}{%
結果重視のフルスタック開発者で、スケーラブルで高性能なWebアプリケーションの構築において実証された専門知識を持っています。現代のJavaScriptエコシステム(React, Vue, Next.js)と堅牢なバックエンドアーキテクチャ(Django, Node.js, NestJS)に特化しています。技術的移行のリード、コードベースの最適化、本番準備の整ったソリューションの提供において実証された成功。コード品質、テスト駆動開発、アクセシビリティ標準の強力な提唱者。現在、データ可視化とユーザーエクスペリエンスの卓越性に焦点を当てたAI駆動型プラットフォームを開発中。
}
\vspace{1pt}

\needspace{8\baselineskip}
\section*{職務経験}

\Experience{AIlon / ERASON}{Lüneburg, Germany}{フルスタック開発者}{2024年10月 -- 現在}
\begin{itemize}
  \item Vue 3、Django REST Framework、Highchartsを使用してエンタープライズクライアントにサービスを提供する本番レベルのAI駆動型分析プラットフォームのアーキテクチャと開発
  \item 大規模データセットを処理する高性能データ可視化コンポーネントのエンジニアリング、最適化技術によりレンダリング時間を60\%削減
  \item プラットフォーム全体で視覚的一貫性を確保する包括的なデザインシステムとコンポーネントライブラリの確立、UI開発時間を40\%削減
  \item Vue 3 Composition API、Pinia、TanStack Queryを使用した高度な状態管理アーキテクチャの実装、アプリケーションの応答性を向上
  \item 製品およびデザインチームとの戦略的意思決定への協力、ロードマップ計画および技術的実現可能性評価への貢献
\end{itemize}

\Experience{Arbithub}{Chicago, USA (Remote)}{フロントエンドWeb開発者インターン}{2024年1月 -- 2024年8月}
\begin{itemize}
  \item ゼロダウンタイムで重要なフレームワーク移行(React 17→18, React Router v5→v6, Redux Toolkit, Tailwind CSS v2→v3)を主導し、バンドルサイズを25\%改善し、開発者エクスペリエンスを向上
  \item 高度なフィルタリング、デバウンス、リアルタイム更新を備えたエンタープライズグレードの検索機能の構築、ユーザーエンゲージメントを35\%増加
  \item ESLintとPrettierによるコード品質標準の確立、技術的負債を40\%削減し、チーム全体で開発ワークフローを標準化
  \item Agile環境での機能提供、クロスファンクショナルチームとのデイリースタンドアップ、スプリント計画、コードレビューへの参加
\end{itemize}

\Experience{LightUp Digital}{Berlin, Germany (Remote)}{フルスタック開発者インターン}{2023年10月 -- 2024年1月}
\begin{itemize}
  \item TypeScript、依存性注入、モジュール設計パターンを使用したNestJSによるスケーラブルなマイクロサービスアーキテクチャの設計と実装
  \item ReactとMaterial-UIを使用した本番準備の整ったレスポンシブインターフェースの開発、開発時間を30\%削減する再利用可能なコンポーネントライブラリの作成
  \item JWTとOAuth 2.0を使用した安全な認証システムのアーキテクチャ、マルチテナントアプリケーションのロールベースアクセス制御(RBAC)の実装
  \item JestとSupertestで85\%のコードカバレッジを達成する包括的なテスト戦略の確立、信頼性と保守性を確保
\end{itemize}

\Experience{Turkish Airlines}{Istanbul, Turkey}{IOCC運用統制官}{2019年3月 -- 2020年5月}
\begin{itemize}
  \item ミッションクリティカルな環境で200以上の日次フライトのリアルタイム運用制御の管理、高圧意思決定とシステム信頼性を必要とする
  \item クロスファンクショナルな国際チームの調整、分散ソフトウェア開発に適用可能な強力なコミュニケーションおよび問題解決スキルの開発
\end{itemize}

\needspace{6\baselineskip}
\section*{注目プロジェクト}

\Project{Hotel Middle Earth — Webアプリケーションゲーム}{%
 没入型のストーリーテリングとリアルタイムマルチプレイヤーインタラクションを備えたインタラクティブなアドベンチャーゲーム \textcolor{verylightgray}{|} HTML、Webアプリケーション、オブジェクト指向プログラミング(OOP)、JavaScript、CSS。ゲームスコアリングとリーダーボード機能のためのFirebaseを使用したバックエンドサーバーアーキテクチャの実装
}

\Project{PlantIQ — インテリジェントな植物ケアアプリケーション}{%
 本番準備の整った植物識別およびケア管理システム \textcolor{verylightgray}{|} JavaScript、Express.js、PlantNet API、Trefle API、Handlebars。リアルタイムの植物健康追跡とパーソナライズされたケア推奨を備えたレスポンシブモバイルファーストインターフェースの構築
}

\Project{React Movie App — エンターテインメント発見プラットフォーム}{%
 高度なフィルタリングとユーザー設定を備えた機能豊富な映画発見アプリケーション \textcolor{verylightgray}{|} React.js、カスタムフック、useReducer、Firebase Auth、Supabase、Cypress E2Eテスト、TMDB API。信頼性と保守性を確保する包括的なテストカバレッジの実装
}

\Project{Memoria — AI強化ソーシャルプラットフォーム}{%
 AI駆動型コンテンツ生成を備えたフルスタックソーシャルネットワーキングプラットフォーム \textcolor{verylightgray}{|} React.js、Node.js、Express、MongoDB、OpenAI API、Tailwind CSS、JWT認証。ARIAラベルとキーボードナビゲーションを備えたアクセシビリティファーストデザインの実装、WCAG 2.1 AA準拠の達成
}

\needspace{5\baselineskip}
\section*{技術スキル}

\noindent
\textbf{\color{darkgray}フロントエンド \& フレームワーク:} React.js, Vue.js, Next.js, TypeScript, JavaScript (ES6+), HTML5, CSS3 \\
\textbf{\color{darkgray}バックエンド \& API:} Python, Django, Django REST Framework, Node.js, Express.js, NestJS, RESTful APIs \\
\textbf{\color{darkgray}データベース \& ORMs:} MongoDB, PostgreSQL, Mongoose, Supabase, データベース設計 \& 最適化 \\
\textbf{\color{darkgray}スタイリング \& UI:} Tailwind CSS, Sass/SCSS, Styled Components, CSS Modules, Bootstrap, Material-UI \\
\textbf{\color{darkgray}状態管理:} Redux Toolkit, Pinia, TanStack Query, Context API, Zustand \\
\textbf{\color{darkgray}テスト \& 品質:} Jest, Cypress, Vitest, Supertest, テスト駆動開発, コードカバレッジ \\
\textbf{\color{darkgray}DevOps \& CI/CD:} Git, GitHub Actions, Docker, Webpack, Vite, ビルド最適化 \\
\textbf{\color{darkgray}クラウド \& サービス:} Firebase, Supabase, Vercel, Netlify, AWS S3, クラウドデプロイメント

\needspace{4\baselineskip}
\section*{主要なコンピテンシー}

\noindent
\begin{minipage}[t]{0.48\textwidth}
  \begin{itemize}
    \item Agile \& Scrum手法
    \item クロスファンクショナルコラボレーション
    \item コードレビュー \& メンタリング
    \item 問題解決 \& デバッグ
  \end{itemize}
\end{minipage}%
\hfill
\begin{minipage}[t]{0.48\textwidth}
  \begin{itemize}
    \item テスト駆動開発(TDD)
    \item パフォーマンス最適化
    \item UI/UXデザイン原則
    \item Webアクセシビリティ(WCAG 2.1)
  \end{itemize}
\end{minipage}

\needspace{6\baselineskip}
\section*{教育}

\Education{フルスタックWeb開発ブートキャンプ}{Ironhack — リモート(400時間集中)}{2023}

\Education{フルスタックWeb開発ブートキャンプ}{Digi-Homeschooling — リモート(400時間集中)}{2022}

\Education{中国語文学士}{Istanbul University, Istanbul, Turkey}{2016 -- 2020}

\Education{社会学学士}{Turkish Military Academy, Ankara, Turkey}{2012 -- 2015}

\section*{言語}

\vspace{1pt}
\noindent
\begin{minipage}[t]{0.24\textwidth}
  \textcolor{darkgray}{\textbf{トルコ語:}} \\
  \textcolor{lightgray}{\small ネイティブ}
\end{minipage}%
\hfill
\begin{minipage}[t]{0.24\textwidth}
  \textcolor{darkgray}{\textbf{英語:}} \\
  \textcolor{lightgray}{\small 完全な職業的}
\end{minipage}%
\hfill
\begin{minipage}[t]{0.24\textwidth}
  \textcolor{darkgray}{\textbf{ドイツ語:}} \\
  \textcolor{lightgray}{\small 完全な職業的}
\end{minipage}%
\hfill
\begin{minipage}[t]{0.24\textwidth}
  \textcolor{darkgray}{\textbf{中国語:}} \\
  \textcolor{lightgray}{\small 限定的}
\end{minipage}

\end{document}
