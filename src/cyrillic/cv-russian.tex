\documentclass[10pt,a4paper]{article}

% Encoding and standard fonts (ATS-friendly)
\usepackage{fontspec}
\setmainfont{Arial Unicode MS}
\setsansfont{Arial Unicode MS}
\renewcommand{\familydefault}{\sfdefault}

% Compact layout for 2 pages
\usepackage[a4paper,margin=1.2cm,top=1.3cm,bottom=1.2cm]{geometry}
\setlength{\parindent}{0pt}
\setlength{\parskip}{1.5pt}
\linespread{1.05}

% Hyperlinks + metadata (ATS-friendly)
\usepackage[hidelinks]{hyperref}
\hypersetup{
  pdftitle={Utku Cikmaz – Full Stack Developer},
  pdfauthor={Utku Cikmaz},
  pdfsubject={Resume, CV, Utku Cikmaz, Full Stack Developer},
  pdfkeywords={Full Stack Developer, React, Vue, JavaScript, Python, Django, Frontend, Backend, AI, Data Visualization, User Experience, Accessibility, Test-Driven Development, Performance Optimization, Cloud Deployment, CI/CD, Git, Docker, Webpack, Vite, Jest, Cypress, Vitest, Supertest, Test Coverage, Code Quality, Agile Methodologies, Scrum, RESTful APIs, Database Design, Microservices, Role-Based Access Control, Multi-Tenant Applications},
  pdfproducer={LaTeX},
  pdfcreator={LaTeX}
}

% Minimal grayscale color palette
\usepackage{xcolor}
\definecolor{darkgray}{HTML}{2D2D2D}     % Dark gray
\definecolor{mediumgray}{HTML}{5A5A5A}   % Medium gray
\definecolor{lightgray}{HTML}{8A8A8A}    % Light gray
\definecolor{verylightgray}{HTML}{B0B0B0} % Very light gray
\definecolor{black}{HTML}{000000}        % Black

% Gradient support for name
\usepackage{tikz}
\usetikzlibrary{shapes,positioning,calc}

% Simple list styling
\usepackage{enumitem}
\setlist[itemize]{
  leftmargin=10pt,
  topsep=1.5pt,
  itemsep=1pt,
  parsep=0pt,
  label=\textcolor{mediumgray}{$\bullet$},
  labelsep=5pt
}

% Clean section styling
\usepackage{titlesec}
\titleformat{\section}
  {\color{darkgray}\large\bfseries\sffamily}
  {}
  {0pt}
  {}
  [\vspace{-4pt}\textcolor{verylightgray}{\titlerule[0.8pt]}\vspace{3pt}]

\titlespacing{\section}{0pt}{8pt}{4pt}

% Prevent section breaks
\usepackage{needspace}

% Compact custom commands
\newcommand{\Experience}[4]{%
  \vspace{2pt}
  \noindent
  \begin{minipage}[t]{0.70\textwidth}
    {\normalsize\textbf{\color{darkgray}#3}} \\
    \textcolor{lightgray}{\small #1}
  \end{minipage}%
  \hfill
  \begin{minipage}[t]{0.28\textwidth}
    \raggedleft
    \textcolor{mediumgray}{\small\textbf{#4}} \\
    \textcolor{lightgray}{\small #2}
  \end{minipage}
  \vspace{1pt}
}

\newcommand{\Education}[3]{%
  \vspace{2pt}
  \noindent
  \begin{minipage}[t]{0.70\textwidth}
    {\normalsize\textbf{\color{darkgray}#1}} \\
    \textcolor{lightgray}{\small #2}
  \end{minipage}%
  \hfill
  \begin{minipage}[t]{0.28\textwidth}
    \raggedleft
    \textcolor{mediumgray}{\small\textbf{#3}}
  \end{minipage}
  \vspace{1pt}
}

\newcommand{\Project}[2]{%
  \vspace{2pt}
  \noindent{\normalsize\textbf{\color{darkgray}#1}} \\
  \textcolor{lightgray}{\small #2}
  \vspace{1pt}
}

% Simple gradient name using varying shades
\newcommand{\gradientname}[1]{%
  \textcolor{darkgray!90}{\Huge\bfseries U}%
  \textcolor{darkgray!80}{\Huge\bfseries T}%
  \textcolor{darkgray!70}{\Huge\bfseries K}%
  \textcolor{darkgray!60}{\Huge\bfseries U}%
  \textcolor{mediumgray!90}{\Huge\bfseries\ }%
  \textcolor{mediumgray!80}{\Huge\bfseries C}%
  \textcolor{mediumgray!70}{\Huge\bfseries I}%
  \textcolor{mediumgray!60}{\Huge\bfseries K}%
  \textcolor{mediumgray!50}{\Huge\bfseries M}%
  \textcolor{lightgray!80}{\Huge\bfseries A}%
  \textcolor{lightgray!70}{\Huge\bfseries Z}%
}

\begin{document}
\color{black}
\pagestyle{empty}

% Minimal header with gradient name
\begin{center}
  \vspace{-3pt}
  \gradientname{UTKU CIKMAZ}
  
  \vspace{4pt}
  \textcolor{verylightgray}{\rule{0.35\textwidth}{0.6pt}}
  
  \vspace{4pt}
  {\large\color{mediumgray}Full Stack Разработчик}
  
  \vspace{5pt}
  \begin{tabular}{c}
    \textcolor{lightgray}{\small Hamburg, Germany} \\[2pt]
    \href{mailto:utkucikmaz@gmail.com}{\textcolor{darkgray}{utkucikmaz@gmail.com}} \quad
    \textcolor{verylightgray}{$\bullet$} \quad
    \href{https://www.utkucikmaz.com}{\textcolor{darkgray}{utkucikmaz.com}} \\[1pt]
    \href{https://linkedin.com/in/utkucikmaz}{\textcolor{darkgray}{linkedin.com/in/utkucikmaz}} \quad
    \textcolor{verylightgray}{$\bullet$} \quad
    \href{https://github.com/utkucikmaz}{\textcolor{darkgray}{github.com/utkucikmaz}}
  \end{tabular}
\end{center}

\vspace{1pt}

\section*{Профессиональный Профиль}
\vspace{0pt}
\textcolor{darkgray}{%
Ориентированный на результат Full Stack Разработчик с доказанной экспертизой в создании масштабируемых, высокопроизводительных веб-приложений. Специализируется на современных экосистемах JavaScript (React, Vue, Next.js) и надежных архитектурах backend (Django, Node.js, NestJS). Продемонстрированный успех в руководстве техническими миграциями, оптимизации кодовых баз и доставке готовых к производству решений. Сильный защитник качества кода, разработки, управляемой тестами, и стандартов доступности. В настоящее время разрабатывает платформы на базе ИИ с фокусом на визуализацию данных и превосходство пользовательского опыта.
}
\vspace{1pt}

\needspace{8\baselineskip}
\section*{Профессиональный Опыт}

\Experience{AIlon / ERASON}{Lüneburg, Germany}{Full Stack Разработчик}{Окт 2024 -- Настоящее время}
\begin{itemize}
  \item Архитектура и разработка производственной платформы аналитики на базе ИИ, обслуживающей корпоративных клиентов с использованием Vue 3, Django REST Framework и Highcharts
  \item Разработка высокопроизводительных компонентов визуализации данных, обрабатывающих крупномасштабные наборы данных, сокращая время рендеринга на 60\% с помощью техник оптимизации
  \item Создание комплексной системы дизайна и библиотеки компонентов, обеспечивающей визуальную согласованность на платформе, сокращая время разработки UI на 40\%
  \item Реализация продвинутой архитектуры управления состоянием с использованием Vue 3 Composition API, Pinia и TanStack Query, улучшая отзывчивость приложения
  \item Сотрудничество с командами продукта и дизайна по стратегическим решениям, вклад в планирование дорожной карты и оценки технической осуществимости
\end{itemize}

\Experience{Arbithub}{Chicago, USA (Remote)}{Frontend Web Разработчик Стажер}{Янв 2024 -- Авг 2024}
\begin{itemize}
  \item Руководство критической миграцией фреймворка (React 17→18, React Router v5→v6, Redux Toolkit, Tailwind CSS v2→v3) с нулевым временем простоя, улучшая размер bundle на 25\% и улучшая опыт разработчика
  \item Построение функциональности поиска корпоративного уровня с расширенной фильтрацией, debouncing и обновлениями в реальном времени, увеличивая вовлеченность пользователей на 35\%
  \item Установка стандартов качества кода с ESLint и Prettier, сокращая технический долг на 40\% и стандартизируя рабочий процесс разработки в команде
  \item Доставка функций в Agile-среде, участие в ежедневных стендапах, планировании спринтов и код-ревью с межфункциональными командами
\end{itemize}

\Experience{LightUp Digital}{Berlin, Germany (Remote)}{Full Stack Разработчик Стажер}{Окт 2023 -- Янв 2024}
\begin{itemize}
  \item Проектирование и реализация масштабируемой архитектуры микросервисов с использованием NestJS с TypeScript, внедрением зависимостей и модульными паттернами проектирования
  \item Разработка готовых к производству адаптивных интерфейсов с React и Material-UI, создание переиспользуемой библиотеки компонентов, сокращающей время разработки на 30\%
  \item Архитектура безопасной системы аутентификации с JWT и OAuth 2.0, реализация контроля доступа на основе ролей (RBAC) для мультитенантного приложения
  \item Установка комплексной стратегии тестирования, достигающей 85\% покрытия кода с Jest и Supertest, обеспечивая надежность и поддерживаемость
\end{itemize}

\Experience{Turkish Airlines}{Istanbul, Turkey}{Офицер Контроля Операций IOCC}{Мар 2019 -- Май 2020}
\begin{itemize}
  \item Управление оперативным контролем в реальном времени более 200 ежедневных рейсов в критической среде, требующей принятия решений под высоким давлением и надежности системы
  \item Координация межфункциональных международных команд, развитие сильных навыков общения и решения проблем, применимых к распределенной разработке программного обеспечения
\end{itemize}

\needspace{6\baselineskip}
\section*{Избранные Проекты}

\Project{Hotel Middle Earth — Игра Веб-Приложение}{%
  Интерактивная приключенческая игра с захватывающим повествованием и взаимодействиями мультиплеера в реальном времени \textcolor{verylightgray}{|} HTML, Веб-Приложения, Объектно-Ориентированное Программирование (OOP), JavaScript, CSS. Реализация архитектуры backend-сервера с Firebase для функциональности подсчета очков игры и таблицы лидеров
}

\Project{PlantIQ — Интеллектуальное Приложение Ухода за Растениями}{%
  Готовая к производству система управления идентификацией и уходом за растениями \textcolor{verylightgray}{|} JavaScript, Express.js, PlantNet API, Trefle API, Handlebars. Построение адаптивного интерфейса mobile-first с отслеживанием здоровья растений в реальном времени и персонализированными рекомендациями по уходу
}

\Project{React Movie App — Платформа Открытия Развлечений}{%
  Функционально богатое приложение для открытия фильмов с расширенной фильтрацией и предпочтениями пользователя \textcolor{verylightgray}{|} React.js, пользовательские хуки, useReducer, Firebase Auth, Supabase, тестирование Cypress E2E, TMDB API. Реализация комплексного покрытия тестами, обеспечивающего надежность и поддерживаемость
}

\Project{Memoria — Социальная Платформа, Улучшенная ИИ}{%
  Полнофункциональная платформа социальных сетей с генерацией контента на базе ИИ \textcolor{verylightgray}{|} React.js, Node.js, Express, MongoDB, OpenAI API, Tailwind CSS, аутентификация JWT. Реализация дизайна, ориентированного на доступность, с метками ARIA и навигацией с клавиатуры, достижение соответствия WCAG 2.1 AA
}

\needspace{5\baselineskip}
\section*{Технические Навыки}

\noindent
\textbf{\color{darkgray}Frontend \& Фреймворки:} React.js, Vue.js, Next.js, TypeScript, JavaScript (ES6+), HTML5, CSS3 \\
\textbf{\color{darkgray}Backend \& API:} Python, Django, Django REST Framework, Node.js, Express.js, NestJS, RESTful APIs \\
\textbf{\color{darkgray}Базы Данных \& ORMs:} MongoDB, PostgreSQL, Mongoose, Supabase, Проектирование Базы Данных \& Оптимизация \\
\textbf{\color{darkgray}Стилизация \& UI:} Tailwind CSS, Sass/SCSS, Styled Components, CSS Modules, Bootstrap, Material-UI \\
\textbf{\color{darkgray}Управление Состоянием:} Redux Toolkit, Pinia, TanStack Query, Context API, Zustand \\
\textbf{\color{darkgray}Тестирование \& Качество:} Jest, Cypress, Vitest, Supertest, Разработка, Управляемая Тестами, Покрытие Кода \\
\textbf{\color{darkgray}DevOps \& CI/CD:} Git, GitHub Actions, Docker, Webpack, Vite, Оптимизация Сборки \\
\textbf{\color{darkgray}Облако \& Сервисы:} Firebase, Supabase, Vercel, Netlify, AWS S3, Облачное Развертывание

\needspace{4\baselineskip}
\section*{Ключевые Компетенции}

\noindent
\begin{minipage}[t]{0.48\textwidth}
  \begin{itemize}
    \item Методологии Agile \& Scrum
    \item Межфункциональное Сотрудничество
    \item Обзор Кода \& Наставничество
    \item Решение Проблем \& Отладка
  \end{itemize}
\end{minipage}%
\hfill
\begin{minipage}[t]{0.48\textwidth}
  \begin{itemize}
    \item Разработка, Управляемая Тестами (TDD)
    \item Оптимизация Производительности
    \item Принципы Дизайна UI/UX
    \item Веб-Доступность (WCAG 2.1)
  \end{itemize}
\end{minipage}

\needspace{6\baselineskip}
\section*{Образование}

\Education{Буткемп Full Stack Web Разработки}{Ironhack — Удаленно (400 часов интенсивно)}{2023}

\Education{Буткемп Full Stack Web Разработки}{Digi-Homeschooling — Удаленно (400 часов интенсивно)}{2022}

\Education{Бакалавр Китайского Языка и Литературы}{Istanbul University, Istanbul, Turkey}{2016 -- 2020}

\Education{Бакалавр Социологии}{Turkish Military Academy, Ankara, Turkey}{2012 -- 2015}

\section*{Языки}

\vspace{1pt}
\noindent
\begin{minipage}[t]{0.24\textwidth}
  \textcolor{darkgray}{\textbf{Турецкий:}} \\
  \textcolor{lightgray}{\small Родной}
\end{minipage}%
\hfill
\begin{minipage}[t]{0.24\textwidth}
  \textcolor{darkgray}{\textbf{Английский:}} \\
  \textcolor{lightgray}{\small Полностью Профессиональный}
\end{minipage}%
\hfill
\begin{minipage}[t]{0.24\textwidth}
  \textcolor{darkgray}{\textbf{Немецкий:}} \\
  \textcolor{lightgray}{\small Полностью Профессиональный}
\end{minipage}%
\hfill
\begin{minipage}[t]{0.24\textwidth}
  \textcolor{darkgray}{\textbf{Китайский:}} \\
  \textcolor{lightgray}{\small Ограниченный}
\end{minipage}

\end{document}
